\def\VCDate{2018/04/25}\def\VCVersion{(Current)}
\documentclass{ffslides}
\ffpage{30}{\numexpr 16/9}
\usepackage{fancyvrb}
\usepackage{ProofPower}
\begin{document}
\obeyspaces

\normalpage{CS124 Lab5 - lab.h}{
    \VerbatimInput[lastline=18]{lab.h} //Don't exceed 20 lines of code per page or else it will run off the slide
}
\ctext{.5}{.1}{.4}{
    \begin{itemize}
        \item Sample page for Humza
        \item This is where the comments go. I formatted it as bullet points. If you don't want bullet points just delete the ```\begin{itemize}``` and ```\end{itemize}``` and type as normal. 
    \end{itemize}
} //end of page

\normalpage{CS124 Lab5 - lab.h}{
    \VerbatimInput[lastline=18]{lab.h}
}
\ctext{.5}{.1}{.4}{
    \begin{itemize}
        \item One third of this header file is literally just include files for the stuff we need from the FLTK library.
    \end{itemize}
}

\normalpage{CS124 Lab5 - lab.h}{
    \VerbatimInput[firstline=20,lastline=29]{lab.h}
}
\ctext{.5}{.1}{.4}{
    \begin{itemize}
        \item I just arbitrarily set the window size to be 400x400
        \item The message struct contains three strings that an email would usually have.
        \item This is for incoming messages.
    \end{itemize}
}

\normalpage{CS124 Lab5 - lab.h}{
    \VerbatimInput[firstline=31]{lab.h}
}
\ctext{.5}{.1}{.4}{
    \begin{itemize}
        \item Lower third of the header file consists of our function prototypes and external variables
        \item We need to have external declarations to declare in other files like our callbacks
    \end{itemize}
}

\normalpage{CS124 Lab5 - BinaryTree.h}{
    \VerbatimInput[lastline=21]{BinaryTree.h}
}
\ctext{.5}{.1}{.4}{
    \begin{itemize}
        \item Class based implementation of the binary search tree
        \item The struct for TreeNode calls the Subscriber struct within it for modular reasons
    \end{itemize}
}

\normalpage{CS124 Lab5 - BinaryTree.h}{
    \VerbatimInput[firstline=24]{BinaryTree.h}
}
\ctext{.5}{.1}{.4}{
    \begin{itemize}
        \item Public member functions include the insertNode, searchNode, remove, and displayInOrder functions
        \item All the functions in here are only applicable to tree objects
    \end{itemize}
}

\normalpage{CS124 Lab5 - Subscriber.h}{
    \VerbatimInput{Subscriber.h}
}
\ctext{.5}{.1}{.4}{
    \begin{itemize}
        \item Self-explanatory struct for subscribers; contains string for name and password.
    \end{itemize}
}

\normalpage{CS124 Lab5 - searchNode.cpp}{
    \VerbatimInput{searchNode.cpp}
}
\ctext{.5}{.1}{.4}{
    \begin{itemize}
        \item Iterative implementation of the search function
        \item If the value of the node is equal to the person we're looking for then we have found the node
        \item If the value is less than the parameter we entered; we go to the left child of the node
        \item Else if the value is greater than the parameter we entered we go to the right child
        \item We keep doing this in the while look until we return true.
    \end{itemize}
}

\normalpage{CS124 Lab5 - remove.cpp}{
    \VerbatimInput[lastline=22]{remove.cpp}
}
\ctext{.5}{.1}{.4}{
    \begin{itemize}
        \item Calls the deleteNode function to remove the root node
    \end{itemize}
}

\normalpage{CS124 Lab5 - makeDeletion.cpp}{
    \VerbatimInput[lastline=22]{makeDeletion.cpp}
}
\ctext{.5}{.1}{.4}{
    \begin{itemize}
        \item TODO!!!!!!!!!!!!!!!!!!!!!!!!!!!!!!!!!!!!!!!!!!!!!!!!!!!!!!!!!!!!!!!!!!!!!!!!!!!!!!!!!!!!!!!!!!!!!!!!!
    \end{itemize}
}

\normalpage{CS124 Lab5 - makeDeletion.cpp}{
    \VerbatimInput[firstline=23]{makeDeletion.cpp}
}
\ctext{.5}{.1}{.4}{
    \begin{itemize}
        \item TODO!!!!!!!!!!!!!!!!!!!!!!!!!!!!!!!!!!!!!!!!!!!!!!!!!!!!!!!!!!!!!!!!!!!!!!!!!!!!!!!!!!!!!!!!!!!!!!!!!
    \end{itemize}
}

\normalpage{CS124 Lab5 - insert.cpp}{
    \VerbatimInput{insert.cpp}
}
\ctext{.5}{.1}{.4}{
    \begin{itemize}
        \item If there are no nodes in the tree then we create the first node i.e. the root.
        \item If the current node we are pointing to has a greater value than the one we are trying to insert then we insert to the left of it.
        \item If the current node we are pointing to has a lesser value than the one we are tryign to insert then we insert to the right of it.
    \end{itemize}
}

\normalpage{CS124 Lab5 - insertNode.cpp}{
    \VerbatimInput{insertNode.cpp}
}
\ctext{.5}{.1}{.4}{
    \begin{itemize}
        \item Allocates a new node to insert
        \item Assigns the data in subscriber to the value of the tree node
        \item Sets the left and right child to null and then calls the insert function
        \item If the insert is successful, we let the user know by producing a message popup
    \end{itemize}
}

\normalpage{CS124 Lab5 - displayInOrder.cpp}{
    \VerbatimInput{displayInOrder.cpp}
}
\ctext{.5}{.1}{.4}{
    \begin{itemize}
        \item Recursive implementation of the display in order function
        \item We are displaying with priority towards the front of the alphabet(A)
    \end{itemize}
}

\normalpage{CS124 Lab5 - deleteNode.cpp}{
    \VerbatimInput{deleteNode.cpp}
}
\ctext{.5}{.1}{.4}{
    \begin{itemize}
        \item Takes two parameters: subscriber address and node pointer
        \item Does comparisons of nodes until correct node is found then calls makeDeletion function
    \end{itemize}
}

\normalpage{CS124 Lab5 - main.cpp}{
    \VerbatimInput[lastline=20]{main.cpp}
}
\ctext{.5}{.2}{.4}{
    \begin{itemize}
        \item The build command is written on the top of the file for reference
        \item Fltk requires the use of many global declarations of each widget you want to create
        \item We hard coded the first account for now due to time constraint and convenience
    \end{itemize}
}

\normalpage{CS124 Lab5 - main.cpp}{
    \VerbatimInput[firstline=22,lastline=34]{main.cpp}
}
\ctext{.5}{.1}{.4}{
    \begin{itemize}
        \item The login buton uses  Fl Button  which is a nice and simple button that will call a callback function when clicked
        \item The name and password fields use  Fl Input  and  Fl Secret Input  respectively for text input
        \item The  Secret Input  function makes it so that your text shows up as censored asterisks when typing in your password
        \item We are able to check for correct name and password in the loginCB callback. We will go into that later.
    \end{itemize}
}

\normalpage{CS124 Lab5 - main.cpp}{
    \VerbatimInput[firstline=36]{main.cpp}
}
\ctext{.5}{.5}{.4}{
    \begin{itemize}
        \item I made the mail window in the main file because...programmer discretion...
        \item This is the window that is supposed to show recieved/incoming mail
        \item There are three buttons currently that can take you to the compose mail window, add subscriber window, and remove subscriber window
    \end{itemize}
}

\normalpage{CS124 Lab5 - loginCB.cpp}{
    \VerbatimInput{loginCB.cpp}
}
\ctext{.55}{.1}{.4}{
    \begin{itemize}
        \item The login callback stores the value of the input fields for name and password and stores them in their own strings
        \item We then make a check if the entered values match with the user's account credentials
        \item If the information entered is wrong then we show a message that alerts the user that the information entered was wrong and allows the user to try again
        \item If the info entered is correct then the user is greeted by the system via a message popup and asked if they want to exit or go to their inbox
        \item If they opt to go to their inbox then we will hide the initial login screen and show the mail window
        \item Pressing the exit button will exit the program
    \end{itemize}
}

\normalpage{CS124 Lab5 - writeMsgCB.cpp}{
    \VerbatimInput[lastline=18]{writeMsgCB.cpp}
}
\ctext{.5}{.1}{.4}{
    \begin{itemize}
        \item The writeMsgCB is the callback for when you click on the "Compose" button and will create a new window which allows for the user to type into the To, Subject, and Message fields
        \item To and Subject uses  Fl Input  to retrieve text and we can use the same method of  $->value()$  to get the input and save to a file with ofstream and ifstream
    \end{itemize}
}

\normalpage{CS124 Lab5 - writeMsgCB.cpp}{
    \VerbatimInput[firstline=20]{writeMsgCB.cpp}
}
\ctext{.57}{.2}{.4}{
    \begin{itemize}
        \item The Message field uses  Fl Text Editor  because it supports multiple lines and has a function for autowrapping text
        \item Important note: when using  Text Editor  you must have a text buffer or else you will not be able to input text into the field
        \item The cancel button calls the cancelMsgCB which we will get to later
        \item At the end, we show the compose message window after all our GUI elements are defined and the actual window has been defined
    \end{itemize}
}

\normalpage{CS124 Lab5 - addSubCB.cpp}{
    \VerbatimInput{addSubCB.cpp}
}
\ctext{.5}{.1}{.4}{
    \begin{itemize}
        \item addSubCB is called when the 'Add Sub' button is clicked and draws up a small window 
        \item This window contains the fields for name and password input using Fl Input
    \end{itemize}
}

\normalpage{CS124 Lab5 - cancelCB.cpp}{
    \VerbatimInput{cancelCB.cpp}
}
\ctext{.5}{.28}{.4}{
    \begin{itemize}
        \item When the cancel button on the 'Add Sub' window is clicked; we alert and ask the user if they are sure
        \item We give them a choice between confirming and declining their action by using fl choice
        \item fl choice returns an integer depending on what the user pressed and we can use that to determine what to do in the switch block
        \item If they choose to decline canceling the sub creation then we just break out and we return back to the 'Add Sub' window
    \end{itemize}
}

\normalpage{CS124 Lab5 - createCB.cpp}{
    \VerbatimInput{createCB.cpp}
}
\ctext{.5}{.1}{.4}{
    \begin{itemize}
        \item We make a Subscriber and BinaryTree object here named sub and tree respectively to pass into the createSub function
        \item However, I worry if I'm actually creating a new binary tree eachtime we want to create a new subscriber
        \item I store the values of the name and password into strings and pass it into createSub
    \end{itemize}
}

\normalpage{CS124 Lab5 - createSub.cpp}{
    \VerbatimInput{createSub.cpp}
}
\ctext{.5}{.25}{.4}{
    \begin{itemize}
        \item The createSub function takes 4 paramenters: name, password, subscriber obj, and tree
        \item This function essentially is just setting all the parameters to the correct places
    \end{itemize}
}

\normalpage{CS124 Lab5 - cancelMsgCB.cpp}{
    \VerbatimInput{cancelMsgCB.cpp}
}
\ctext{.53}{.1}{.4}{
    \begin{itemize}
        \item When the cancel button on the 'Compose' window is clicked; we alert and ask the user if they are sure
        \item This process is extremely similar to the cancelCB function for the subscribers screens. We could probably make this code reusable.
    \end{itemize}
}


\normalpage{Build Script}{

\begin{GFT}{Text written to file build.sh}
\+pptexenv latex lab\\
\+dvipdf lab\\
\end{GFT}
\begin{GFT}{Bourne Shell}
\+chmod +x build.sh\\
\end{GFT}
}
\end{document}
